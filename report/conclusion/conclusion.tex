\chapter{Conclusion}

This project began with the RobotWeb and a few ideas on how to attack it, but none on how to defend it against these attacks. Throughout the duration of this project, we have set and achieved several sub-objectives. First, we provided a thorough analysis of the security vulnerabilities of the RobotWeb (\autoref{section:attack-analysis}), using this analysis to predict and verify the emergence of six behaviours (\autoref{section:6hyp} and \autoref{section:attacker_experiments}). Armed with this knowledge we set out to design and implement Aegis, an algorithm to defend the RobotWeb (\autoref{section:aegis}), leaving many imperfect solutions by the wayside (\autoref{section:indiv-limits}). Once we arrived at a final design for Aegis, we proved its correctness (\autoref{section:proof-correctness}) and then experimentally verified its validity for our implementation of Aegis (\autoref{section:eval-def}). Finally, we explored some techniques to tune the performance of Aegis, illuminating avenues for further research. 

\section{Future work}
\begin{center}
We conclude this thesis with an exposition of future work. 
\end{center}

\subsection{Real World Testing}
The most practical avenue of future work here would be to test Aegis in real-world scenarios. So far we have only had access to a RobotWeb simulator, but are aware of ongoing efforts to implement a physical RobotWeb using real robots. It then naturally follows that Aegis should be tested within it.

Testing Aegis in real-world scenarios will likely open a Pandora's Box of optimisation opportunities as its performance in real-world settings is likely to depend on physical characteristics not found in simulation. For example, our simulations assume that robots will move in 2-dimensional space when in reality this is impossible.

It would also be desirable to test Aegis in a truly heterogeneous setup; where some robots may fly whilst others move on the ground, where each robot carries different sets of sensors each with different capabilities and where each robot runs asynchronously from its peers.

\subsection{Further Optimisation of Aegis' Policies}
Another interesting avenue of future work would be to further optimise the Successor Identification and Retirement Policies used by Aegis. In \autoref{section:eval-def}, we laid the groundwork for this by introducing and evaluating some simple policies. Yet as we discussed, there is further scope for improving these, for example, hybrid policies could be tested, or reinforcement learning could be used to search for optimal policies. As we showed, the choice of policy does impact the overall performance of Aegis, thus optimising them may lead to substantial improvements.


\subsection{Privacy Preservation}
Privacy preservation is also another promising avenue of future work. Although Aegis prevents attackers from seizing control of other robots' localisations, it does not preserve their privacy. Robots constantly send messages with their current localisations, which creates privacy concerns as nefarious agents may use these.

We believe that an interesting starting point for this would be to use indirection. Instead of sharing its localisation, and its beliefs of others' localisations, a robot would share its belief of the localisation of some shared point $p$. Other robots would also measure $p$ and compare their belief of its localisation with the robots, using these to update their own beliefs. This technique would ensure that robots never publish their localisations, but instead those of their surroundings.

\subsection{Path Planning}
Work has been done to extend the principles of the RobotWeb to allow for path planning \cite{pathplanning}. Yet this also extends the vulnerabilities of the RobotWeb to the path planning algorithm. If this were to be attacked, then the attacker would have even greater control of other robots than if it were to merely control their localisations. It is for this reason that we believe that the principles of Aegis should potentially be extended to also cover path planning.

\subsection{General Purpose Sensor Networks}
Although much of this research has focused on the field of robotics, we believe that it should be possible to apply Aegis to other systems that use Gaussian Belief Propagation. These systems are most likely sensor networks.