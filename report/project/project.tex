\chapter{Project Plan}

\section{Current State}
Since this thesis involves extending the Robot Web, I'll briefly explain how the current codebase works. The codebase simulates a set of agents moving along fixed paths, in the Robot Web. It takes a configuration file as an input, which defines how many agents should exist, what the paths should be, where beacons should be placed and how much noise exists in the system. The simulator can also run in headless mode, where it logs each robot's estimated pose against time, which can be used to calculate summary statistics.

The codebase was originally designed to work with identical ``agents'' (robots) moving along fixed paths. Since attackers weren't present, the code was able to use a single connected factor graph, rather than a distributed set of fragments, which was a problem as it didn't make it easy for agents to control their message passing. The consequence of this was that it made it harder to implement attacks and defences. 

First, I changed the code to allow different types of agents to exist within a configuration. Then I refactored the code to distribute fragments of the factor graph to each agent, which would allow them to attack others/defend themselves by taking control of which messages they send and receive. I've also added a host of different configurable tracks for the robots to move along, which can be seen below. I think will help with experimentation later in the project.

I've also implemented some basic attacks on the system. These include:
\begin{enumerate}
    \item  The ``overconfidence attack'' where an attacker sends a bad reading with a tiny standard deviation.
    \item The ``spoofing attack'' where an attacker sends these measurements but uses the ID of a different agent.
    \item The Sybil attack, where an attacker sends bad readings from many phantom agents. An interesting side note here is that the readings for the Sybil attack don't need to modify the standard deviation in the messages, which means that victims won't be able to easily distinguish them from correct messages.
\end{enumerate}
Right now these attacks are in a basic state since they don't try to defeat any countermeasures that can be used.

Finally, I've written an experiment runner script, which would generate several configurations, run the simulation using them, and graph the results. This is mainly to automate experimentation.

\section{Milestones}
\newcommand{\foo}{\hspace{-2.3pt}$\bullet$ \hspace{5pt}}

\scalebox{1}{
\begin{tabular}{r |@{\foo} l}

Mid-Feb 2023 & Fully finish refactoring taking into account feedback from an early January meeting\\
End-Feb 2023 & Finalise some ideas on how to defend against all attacks on robot localisation\\
Mid-Mar 2023 & Implement the defenses\\
End-Mar 2023 & Exam revision\\
Mid-Apr 2023 & Extend RobotWeb to allow non-robot localisation\\
End-Apr 2023 & Implement/Extend defences for this\\
Mid-May 2023 & Run experiments on physical robots\\
End-June 2023 & Write report.

\end{tabular}
}

\section{Stretch Goals}
The main stretch goal I have in mind is to further extend the security to allow robots to share their path information. This will be harder to do than the other defences, as it is only possible to find out if a robot has lied in the future when it either takes or doesn't take the path.