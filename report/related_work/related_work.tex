% \chapter{Related Work}
\noindent\makebox[\linewidth]{\rule{\textwidth}{1pt}}


Several different techniques have been explored in preventing the types of identity-based attacks discussed in this chapter. In the \nth{2} half of this chapter, we will examine and evaluate these. We broadly group these approaches into 2 main groups; the first uses the physical characteristics of signal propagation to bind an identity to an entity, whilst the second exploits the fact that no entity can perform an unlimited amount of computation.

\section{Wifi Fingerprinting}
% In this section, we will review two Wifi fingerprinting-based approaches to binding identities to entities. 

There are many ways for devices to communicate wirelessly, many of which use radiowaves. Wifi is the name of a family of networking protocols that allow for this, it derives from the IEEE 802.11 standard. In order to use Wifi, a device must have at least one wireless antenna which can transmit and receive within the bands specified in the IEEE 802.11 standard; usually 2.4GHz and 5GHz.

When a device sends a packet using Wifi, it transmits a radio signal for a given number of nanoseconds from its antenna. This signal will attenuate as it travels further and further through space. After a certain distance, also known as the communicating range of the antenna, the signal will fade into background noise. The signal leaves the antenna in all directions simultaneously, as a radio wave. Eventually, a small part of this wave will reach the receiver's antenna and the packet will be decoded out of it.

Other parts of the wave will either diffract around corners, reflect off some large objects, or scatter off many small objects \cite{SignalProp}. Consequently, this means that a receiver may receive a packet from several different directions simultaneously, that is that it could encounter different parts of the same wave from different directions at the same time. This phenomenon is known as multipath scattering. %TODO Add diagram? 

Multipath scattering has two interesting properties; it is practically impossible to predict, ahead of time, the distribution of signals around a receiver and it is unlikely that two receivers will observe the same signal propagation with sufficient multipath scattering\citationneeded. These properties allow for devices to ``fingerprint'' every packet they receive, such that two different transmitters cannot have the same fingerprint unless they are simultaneously located at the same place. 

However, implementing multipath scattering-based algorithms have some technical constraints; namely that the receiving antenna must be able to measure the signal in each direction, and that a sufficient amount of multipath scattering must occur. Usually, these algorithms struggle in outdoor environments, where the environment may not provide objects for multipath scattering to occur\citationneeded.

The following papers discuss methods to circumvent these constraints, focussing on securing robot networks from spoofing and Sybil attacks. 

\subsection{Guaranteeing spoof-resilient multi-robot networks}
Gil et al, \cite{GuaranteeingSpoofResilience} provides an interesting approach to some of the aforementioned problems, most notably the problem of requiring expensive hardware to allow the receiving antenna to measure the signal in each direction. They do this by inventing an algorithm that allows them to build a ``virtual spoofer sensor'' only using commercially available wifi hardware, which creates a ``spatial fingerprint'' from each transmission in the network. They use the output from this sensor to calculate a confidence metric $\alpha$ indicating their algorithm's confidence that a robot's identity is its entity. Finally, they characterise the theoretical performance of the ``virtual spoofer sensor'' and provide empirical evidence to support their claims by undertaking several experiments.

The authors build the ``virtual spoofer sensor'' by building upon \textit{Synthetic Aperture Radar} (SAR) techniques, which allow a single antenna to be used to simulate an antenna array. SAR involves moving the antenna to different locations and taking snapshots of the signals received. These snapshots are then combined using signal-processing techniques to emulate a multi-antenna array \cite{BAD_SAR}.\unsure{Diagram needed?}
%TODO: This isn't a great citation, go to the original paper
The ``spatial fingerprint'' calculated, is then compared to the fingerprints of other clients, and clients with identical fingerprints are assumed to be Sybil attackers.

The authors evaluate their algorithm in the context of the following problem statement. Given an environment with several ``clients'', each expecting service from mobile ``servers'', dynamically find the optimal layout for the servers such that each ``client'' is served. A subset of clients are assumed to be malicious and are carrying out Sybil attacks in order to influence the ``servers'' to move closer to them.

The authors perform four experiments to validate their hypotheses:
\begin{enumerate}
    \item They compare the performance of the ``virtual spoofer sensor'' in both an indoor and simulated outdoor environment, as expected, finding that multipath scattering is more effective in indoor settings, but also that adding a single reflector to the environment vastly improves performance.
    \item They compare the effect of a stationary, moving, and power-scaling Sybil attacker on the ability of the ``virtual spoofer sensor'' to correctly classify agents, resulting in no false negatives, but many false positives.
    \item They evaluate their system on the multi-agent coverage problem \cite{MultiAgentCoverage}, finding that it can provide near-optimal results even when there are 3$\times$ more spoofed agents.
    \item They apply their system to a drone delivery problem, where the ``server'' needs to visit each real ``client'' to deliver a package and again find that their system provides near-optimal results when there are 3$\times$ more spoofed agents.
\end{enumerate}

\unsure{I have suspicions about how they did the last 2 experiments, especially since the phantoms are nowhere near the actual attacker.}

\subsection{Lightweight Sybil-Resilient Multi-Robot Networks by Multipath Manipulation}
Huang et al. \cite{MultiPathManipulation} take a different approach to Gil et al. \cite{GuaranteeingSpoofResilience}; instead of relying upon the environment to provide multipath scattering, they actively cause it by using backscatter tags. This offers two main advantages over Gil et al.: \begin{enumerate*}
    \item the environment has a lesser effect on the performance of the Sybil attack detector and
    \item the robots' antennae no longer need to move when capturing a fingerprint.
\end{enumerate*}
Using the captured signal information, the robots again compute a fingerprint per transmission, normalise it to mitigate the effects of any power scaling attacks, and finally compare the normalised fingerprint to those of all others, treating any identities with sufficiently similar fingerprints as Sybil attackers.

Backscatter tags scatter signals that they encounter by rapidly absorbing and reflecting them. Backscatter tags also simplify the fingerprinting process, since robots are no longer required to perform small movements for SAR, nor must the software engage in expensive linear algebra to construct a multi-antenna array. 

A key property of backscatter tags is that they operate passively and don't require a power supply. This reduces the cost of implementing this scheme as tags can be simply and inexpensively attached to robots. Another useful property is that the backscattering of the final signal is highly correlated with the positions of the tags, transmitting and receiving antennae, meaning that if two identities have very similar backscattering patterns, they are likely to originate from the same entity.

When a robot receives a signal, it receives a raw signal, which is first smoothed out with a moving average filter, to create the backscattered signal. Then it decodes a message out of the backscattered signal, and uses it, with signal processing techniques to deduce how much each tag contributed to the backscattering, and when each tag was ``activated''. Then the robot does more filtering to remove any backscattering from the environment, the resulting signal will be used to construct a signature for the transmission.

The authors then evaluate their implementation in both an indoor office environment and an outdoor rooftop environment. They find that their method is virtually indifferent to the surrounding environment, as they measure an average true positive rate of 97.6\% and an average false positive rate of 5.1\%.

\subsection{Conclusions}

Although both sets of authors extensively test their system, they make some problematic assumptions, which may be exploited by attackers. 

\begin{enumerate}
    \item They do not account for Sybil attacks using multiple antennae, which could transmit the same message at the same time, but with variable power levels. Each antenna would create a different multipath scatter, and if the relative powers between them were varied, then they could theoretically construct many false fingerprints.
    \item They also do not account for collaboration between different attackers, which would function similarly to the previous vulnerability, where geographically distributed attackers could simultaneously send the same message, with different power scales, creating another set of false fingerprints. This method could produce a larger range of false identities but may encounter synchronisation problems between attackers.
    \item Attackers could leverage methods similar to Huang et al. and physically augment their antennae to manipulate their multipath scattering, for example, one could create moveable barriers to prevent some scattering from occurring.
    \item Both sets of authors assume that any noise from the environment will not be malicious; Gil et al. assume that it will follow a Gaussian distribution, whilst Huang et al. assume that standard signal processing techniques would be sufficient to filter it out. Both of these assumptions fail to account for the possibility that coordinated attackers emit noise above background levels, but not so high that it would seriously interfere with transmissions. This would disrupt every transmission's signature and would prevent any pair of transmissions from looking alike.
\end{enumerate}


% \todo{Add diagram for 3}

\section{Proof of Work}
Proof of Work is underpinned by the insight that no entity in a network can perform unlimited computation. This leads to defences reliant on the idea that generating identities should be computationally expensive, to prevent any entity from being able to create an unbounded number of them. A PoW identity generation scheme would express an identity as the solution to some puzzle, and thus the identity can be validated by checking if it is a valid solution to the puzzle. This imposes two constraints: \begin{enumerate*}
    \item the puzzle must be hard to solve and
    \item the solution to a puzzle must be easy to verify
\end{enumerate*}, which means that, formally, the puzzle belongs to the \textit{NP-Complete} class of problems.

A problem with this strategy is that it requires wasting computational resources, which may be in short supply for the embedded systems used in robotics. Furthermore, it requires that normal robots constantly pay a steep price for their security even without the presence of attackers.

Gupta et al. \cite{PoW} design an iterative algorithm (\textbf{GMCom}) that solves the \nthM{2} problem, where if attackers spend $T$ resources, whilst $J$ new non-attacking identities are presented, then each non-attacker only needs to spend $O(\sqrt[2]{TJ} + J)$ resources. They refer to this property as the \textit{assymetry} of the algorithm, as attackers must spend many more resources than non-attackers.

GMCom organises identities into a group, where a subset of them form a committee. When a new identity tries to join the group, it must first solve a puzzle set by the committee. Occasionally, the committee will seek to purge all attackers from the group, by issuing a \textit{purge puzzle}, which must be solved by all identities before a new committee is formed, otherwise the identity will be purged from the group. This limits the wasted computation that good entities must perform since they only need to solve a single puzzle when entering a group, and occasionally thereafter, whilst an attacker would need to solve puzzles for each identity it claims, and would need to solve them repeatedly to avoid being purged.

However, this approach is not without its limitations. The authors assume that attackers will always only be able to command a fraction, $\alpha$, of the total resources of the system, however, the heterogeneous nature of robots means that this may not always be guaranteed. For example, in a drone delivery system, each drone would likely only possess a small amount of computational power, but an attacker may attack the system using their desktop computer.

% GMCom organises identities into a group, where a subset of them form a committee. When a new identity tries to join the group, it must first solve a current epoch's puzzle, set by the committee, and broadcast its solution. Once a certain number of identities has joined or left the group, the committee starts the next epoch with a new puzzle and new committee. 
% \section{Attacks on sensor networks}
% \section{Reputation based systems}
