\chapter{Introduction}
\section{Objectives}

% The field of distributed robotics is a growing one, partly due to the growth of the number of applications involving it (autonomous vehicles and drone delivery systems) and partly because of increased interest in areas which would greatly benefit from it, such as the Lunar Gateway Project, or the Mars 2020 Perseverance Mission.

% An open problem in this distributed robotics is that of effective inter-robot communication. Inter-robot communication provides several benefits to distributed robotics; it allows robots to
% \begin{enumerate*}
%     \item acquire a more accurate view of their environment,
%     \item gain access to a larger map of the world,
%     \item improve their path planning by incorporating others' plans and
%     \item carry fewer resources, such as sensors and instead rely on others in the swarm.
% \end{enumerate*}

% A subset of distributed robotics research is dedicated to finding ways to allow such communication in situations where some of the robots may act with hostility. The source of this hostility could be unnatural, where a bad actor solely seeks to disrupt the system, or it could be a natural consequence of competition in the environment, such as a self-driving car trying to prevent others from overtaking it. 

% This hostility must be protected against if the benefits of communication and collaboration are to be seized. This is the principal focus of this Master's Thesis.

In the future, there will be more robots. More robots on our roads, in our skies and our homes, more robots on the Moon and more robots on Mars. Looking at today's world, we can already catch a glimpse of tomorrow's - we see a growing number of self-driving cars, of drones and of Roombas. Eventually, these nascent technologies will become ubiquitous. 

Now the question is not whether these robots will talk to one another, but how. They will talk to get a better understanding of where they are, to learn about all the places they cannot see, or to avoid bumping into each other. Talking will also allow individual robots to specialise and carry fewer sensors, instead relying upon their peers for some sensor readings. This interconnectedness would place the robots in their own pseudo-society.

As with all societies, ours will need a way to deal with conflicts between members. Conflicts may arise in the most innocent of circumstances, such as when two cars both want to take the same parking space. But, they may also arise when a few bad actors choose to act with hostility and misinform others.


If we don't find a way to handle hostility, we will lose all the benefits of communication. Why would a robot choose to rely on others' information if it thinks they would mislead it? Why would a robot provide information to others if it believes they will use it against it? Robots would instead return to an individualistic worldview, increasing the number of sensors they'd need to carry, the amount of computation they'd need to perform and limiting their knowledge of the world around them. 

\centerline{The principal focus of this Master's thesis is to resolve this problem.}

\section{Challenges}
\section{Contributions}
Currently, the main contributions of this paper can be found in the background research section and some of the preliminary results on the effectiveness of attacks on robot networks. This is subject to change with time.