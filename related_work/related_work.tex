\chapter{Related Work}

Several different techniques have been explored in preventing the types of identity-based attacks discussed in the previous chapter. In this chapter, we will examine and evaluate these. We broadly group these approaches into 2 main groups; the first uses the physical characteristics of signal propagation to bind an identity to an entity, whilst the second exploits the fact that no entity can perform an unlimited amount of computation.

\section{Wifi Fingerprinting}
% In this section, we will review two Wifi fingerprinting-based approaches to binding identities to entities. 

There are many ways for devices to communicate wirelessly, many of which use radiowaves. Wifi is the name of a family of networking protocols that allow for this, it derives from the IEEE 802.11 standard. In order to use Wifi, a device must have at least one wireless antenna which can transmit and receive within the bands specified in the IEEE 802.11 standard; usually 2.4GHz and 5GHz.

When a device sends a packet using Wifi, it transmits a radio signal for a given number of nanoseconds from its antenna. This signal will attenuate as it travels further and further through space. After a certain distance, also known as the communicating range of the antenna, the signal will fade into background noise. The signal leaves the antenna in all directions simultaneously, as a radio wave. Eventually, a small part of this wave will reach the receiver's antenna and the packet will be decoded out of it.

Other parts of the wave will either diffract around corners, reflect off some large objects or scatter off many small objects \cite{SignalProp}. Consequently, this means that a receiver may receive a packet from several different directions simultaneously, that is that it could encounter different parts of the same wave from different directions at the same time. This phenomenon is known as multipath scattering. %TODO Add diagram? 

Multipath scattering has two interesting properties; it is practically impossible to predict, ahead of time, the distribution of signals around a receiver and it is unlikely that two receivers will observe the same signal propagation with sufficient multipath scattering \citationneeded. These properties allow for devices to ``fingerprint'' every packet they receive, such that two different transmitters cannot have the same fingerprint unless they are simultaneously located at the same place. 

However, implementing multipath scattering-based algorithms have some technical constraints; namely that the receiving antenna must be able to measure the signal in each direction, and that a sufficient amount of multipath scattering must occur. Usually, these algorithms struggle in outdoor environments, where the environment may not provide objects for multipath scattering to occur \citationneeded.

The following papers discuss methods to circumvent these constraints, focussing on securing robot networks from spoofing and Sybil attacks. 

\subsection{Guaranteeing spoof-resilient multi-robot networks}
Gil et al, \cite{GuaranteeingSpoofResilience} provides an interesting approach to some of the aforementioned problems, most notably the problem of requiring expensive hardware to allow the receiving antenna to measure the signal in each direction. They do this by inventing an algorithm that allows them to build a ``virtual spoofer sensor'' only using commercially available wifi hardware, which creates a ``spatial fingerprint'' from each transmission in the network. They use the output from this sensor to calculate a confidence metric $\alpha$ indicating their algorithm's confidence that a robot's identity is its entity. Finally, they characterise the theoretical performance of the ``virtual spoofer sensor'' and provide empirical evidence to support their claims by undertaking several experiments.

The authors build the ``virtual spoofer sensor'' by building upon \textit{Synthetic Aperture Radar} (SAR) techniques, which allow a single antenna to be used to simulate an antenna array. SAR involves moving the antenna to different locations and taking snapshots of the signals received. These snapshots are then combined using signal-processing techniques to emulate a multi-antenna array \cite{BAD_SAR}. \unsure{Diagram needed?}
%TODO: This isn't a great citation, go to the original paper
The ``spatial fingerprint'' calculated, is then compared to the fingerprints of other clients, and clients with identical fingerprints are assumed to be Sybil attackers.

The authors evaluate their algorithm in the context of the following problem statement. Given an environment with several ``clients'', each expecting service from mobile ``servers'', dynamically find the optimal layout for the servers such that each ``client'' is served. A subset of clients are assumed to be malicious and are carrying out Sybil attacks in order to influence the ``servers'' to move closer to them.

The authors perform four experiments to validate their hypotheses:
\begin{enumerate}
    \item They compare the performance of the ``virtual spoofer sensor'' in both an indoor and simulated outdoor environment, as expected, finding that multipath scattering is more effective in indoor settings, but also that adding a single reflector to the environment vastly improves performance.
    \item They compare the effect of a stationary, moving and power-scaling Sybil attacker on the ability of the ``virtual spoofer sensor'' to correctly classify agents, resulting in no false negatives, but many false positives.
    \item They evaluate their system on the multi-agent coverage problem \cite{MultiAgentCoverage}, finding that it can provide near-optimal results even when there are 3$\times$ more spoofed agents.
    \item They apply their system to a drone delivery problem, where the ``server'' needs to visit each real ``client'' to deliver a package and again find that their system provides near-optimal results when there are 3$\times$ more spoofed agents.
\end{enumerate}

\unsure{I have suspicions about how they did the last 2 experiments, especially since the phantoms are nowhere near the actual attacker.}

Although the authors extensively test their system, they make some problematic assumptions which may be exploited by attackers. Firstly they do not account for Sybil attacks using multiple antennae, which could transmit the same message at the same time, but with variable power levels. Each antenna would create a different multipath scatter, and if the relative powers between them were varied, then they could theoretically construct many false fingerprints. Similarly, they do not account for collaboration between different attackers, which would again be able to create many false fingerprints. Thirdly, attackers could physically augment their antennae to partially control their multipath scattering, for example, one could create moveable barriers to prevent some scattering from occurring. Finally, the authors assume that all background noise will follow a Gaussian distribution, however, this assumption falls flat in an adversarial scenario as other attackers could emit non-Gaussian noise to decrease the similarity of fingerprints.
\todo{Add diagram for 3, and maybe make this an enumeration}


\subsection{Lightweight Sybil-Resilient Multi-Robot Networks by Multipath Manipulation}
Huang et al.

Explain how they generate a fingerprint and how they use it to combat sybil and masquerade attacks

Present their findings

\subsection{Conclusions}

Talk about what's wrong with the papers/what can be done to break the attack

\section{Proof of Work}
Explain how proof of work usually works (blockchain context)

\subsection{Paper series 1 - Gupta et al.}
Explain what problems w PoW they find

Explain how they solve them and the results

\subsection{Paper series 2 - Strobel et al.}

Explain what problems w PoW they find

Explain how they solve them and the results

\subsection{Conclusions}

\section{Attacks on sensor networks}

