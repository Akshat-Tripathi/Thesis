\chapter{Introduction}
\section{Objectives}

The field of distributed robotics is a growing one, partly due to the growth of the number of applications involving it (autonomous vehicles and drone delivery systems) and partly because of increased interest in areas which would greatly benefit from it, such as the Lunar Gateway Project, or the Mars 2020 Perseverance Mission.

An open problem in this distributed robotics is that of effective inter-robot communication. Inter-robot communication provides several benefits to distributed robotics; it allows robots to
\begin{enumerate*}
    \item acquire a more accurate view of their environment,
    \item gain access to a larger map of the world and
    \item improve their path planning by incorporating others' plans.
\end{enumerate*}

A subset of distributed robotics research is dedicated to finding ways to allow such communication in situations where some of the robots may act with hostility. The source of this hostility could be unnatural, where a bad actor solely seeks to disrupt the system, or it could be a natural consequence of competition in the environment, such as a self-driving car trying to prevent others from overtaking it. 

This hostility must be protected against if the benefits of communication and collaboration can be seized. This is the principal focus of this Master's Thesis.


\section{Challenges}
\section{Contributions}
Currently, the main contributions of this paper can be found in the background research section and some of the preliminary results on the effectiveness of attacks on robot networks. This is subject to change with time.