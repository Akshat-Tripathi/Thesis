\chapter{Conclusion}

\section{Ethical Considerations}

\begin{quote}
    \centering 
    ``A new device merely opens a door; it does not compel one to enter''\\
    \attrib{Lynn White \cite{MedievalTechnology}}
\end{quote}

From the discovery of coffee leading to the ``Age of Enlightenment'' to the invention of Boolean algebra creating our modern digital age, history has repeatedly shown us that it is impossible to fully understand the implications of new discoveries and nascent technology. With this in mind, we provide a short discussion of potential ethical issues which, we believe, may arise as a consequence of the research conducted in this thesis.

As with all research enabling autonomous robotics, we must consider potential military applications. Many militaries today already use unmanned aerial combat vehicles in their operations, if they were to incorporate this research, they may be able to improve their performance by allowing them to share information securely. However, we do not believe that this is likely to occur as militaries tend to have highly centralised structures, where each robot would have prior knowledge about other trusted robots in the network. Whereas our research focuses on providing security to untrusted robots in decentralised networks.

Another potential misuse of this research would be enhancing the capabilities and security of surveillance robots. In this scenario, an authoritarian regime would use robots to continuously monitor their citizens. The robots would communicate with others in their immediate surroundings to coordinate their search and could be vulnerable to cyber attacks where several are hijacked. The hijacked robots would then send incorrect messages to prevent certain areas from being searched. However, it is unlikely that this research would be an ideal candidate for implementing such a dystopia, as a single party would own the robots and would find it much simpler to implement centralised security measures.

Alongside the unethical misuses of this research, there exist several scenarios where it would confound unethical groups. One intended use case is to implement a common robotic infrastructure for autonomous robots owned by many different parties. Here the decentralised nature of the infrastructure would provide asymmetric robustness against hackers or governments seeking to unilaterally disrupt and destroy the infrastructure as they would not be a single point of failure.

In conclusion, there are many scenarios where this research may be misused to the detriment of humanity, yet we are not convinced that this research would be the most appropriate in those examples. Furthermore, given how this research seeks to defend against bad actors, we believe it is more likely to be a benefit to humanity.